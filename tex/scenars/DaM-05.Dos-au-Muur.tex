\section{Dos au Muur}
On attaque ici le cinquième et dernier chapitre de la saga.

\subsection{Résumé des épisodes précédents}
Le camp oposé (l’empire pour les gentils et l’alliance si vos héros sont avec Dark Vador) a récupéré l’\nameref{sec:oubliette-de-dreypa} au nez et à la barbe de vos héros qui sont donc bien dégouttés. Possiblement, vos héros ont trouvé l'Holocron de Muur dans le laboratoire de \nameref{sec:pulsipher} mais pas forcément. Enfin, vos héros savent ce que contient l’oubliette.

Vos héros vont donc errer un moment dans l’Univers à la recherche d’informations qui pourraient les mener à l’oubliette perdue. Selon que vos joueurs se montrent imaginatif ou pas sur la recherche, abrégez ou non la séquence par un appel de leur hiérarchie qui aura une info à leur faire part. Si vos joueurs se montrent imaginatif faites leur trouver la dite information par eux mêmes.

\subsection{Sur la piste}
\paragraph{Empire}
Si vos héros sont à la solde de l’Empire, l’information qui finie par leur parvenir est que des espions infiltrés ont découvert au péril de leur vie que les rebels ont améné sur une base reculée de l’alliance, une cargaison très spéciale et que depuis seul le personnel fortement accrédité est autorisé à circuler dans la zone de stockage de la cargaison.

La base rebelle se trouve sur la lune IV de Yavin. Les rebels sont établi dans ce système depuis plusieurs mois. La quatrième lune de Yavin est une lune forestière avec une végétation dense offrant un bon camouflage aux batiments installés sous la canopée. La lune possède plusieurs installations distante les unes des autres de plusieurs centaines de kilomètres. La fameuse cargaison est entreposée et sans doute étudiée dans l’installation la plus au Nord et la plus éloignée de toute les autres.

\paragraph{Alliance}
Si vos héros se sont enrolés dans la résistance, c’est un peu la même histoire. L’information qui leur parvient est que l’Empire à fait porter sur une lune de Lothal une cargaison et que depuis le batiment où est entreposée la cargaison est verrouillé. 

Sur l’une des lunes de Lothal, l’Empire possède une base militaire. La lune est de type désertique. Pas de végétation si d’eau, seulement une atmosphère un peu raréfiée. La cargaison est entreposée 100~km au Nord de la base dans un zone qui depuis est fermée à tous les soldats non autorisés.

\paragraph{Commun}
Dans tous les cas, les héros sont prévenu que le camp adverse s’emploit à ouvrir l’Oubliette mais que pour l’instant il n’y est pas parvenu.

Donc là normalement vos héros sentent (ou non) le piège, il leur faut donc un plan. Ils savent que dans l’oubliette se trouve \nameref{sec:celeste-morne}, une puissante Jedi et, qui plus est, possédée par le Talisman de Muur, un non moins puissant Sith ! Le combat, si combat il y a, s’annonce un peu difficile. A vous de leur faire comprendre, s’ils ne le ressentent pas spontanément, qu’y allez sans un plan pour libérer Céleste du Talisman n’est rien de plus qu’un suicide collectif.

\subsection{Man with a plan}
C’est là qu’intervient l’Holocron de Muur.

S’ils n’ont pas trouvé, et bien c’est qu’ils ont mal joué ! La meilleure solution est encore de leur faire comprendre qu’il leur manque une pièce du puzzle et qu’ils doivent retourner sur \nameref{sec:jebble} chercher l’Holocron. Par exemple on leur rappelant que le datapad de \nameref{sec:pulcipher} faisait référence à un object triangulaire.

S’ils ont l’Holocron avec eux, il est temps de leur faire comprendre que c’est un élément important de l’histoire et qu’il va falloir trouver comment l’ouvrir. Pour l’ouvrir ils vont devoir retourner sur \nameref{sec:taris} dans les ruines de l’ancien temple Sith. Sur le Monolithe en y regardant d’un peu plus prêt on trouvera un emplacement triangulaire où l’Holocron entre à la perfection. Cette fois, on leur facilite le passage, pas de gros boss ni de complication particulière. C’est comme dans les jeux une fois que la zone est visitée elle est sécure. Et puis l’objectif de la mission n’est pas de refaire le scénario 2.

\newpage
\begin{paperbox}{Comment les amener sur Taris ?}
Quelques idées sur comment les amenà à retourner sur Taris pour ouvrir l’Holocron. Car ce dernier étant verrouillé sur l’esprit de Muur, il n’est pas possible de l’ouvrir, même pour une apprenti Sith.

    \begin{rebelist}
        \item S’il y a un héros sensible à la Force parmis les joueurs, le plus simple et de lui donner une vision dans laquelle il voit le monolithe. Ou plus subtil, il voit les mêmes évènements que ceux décrit pas le Monolithe.
        \item Sinon, \nameref{sec:garan-keggle} ou \nameref{sec:dass-jennir} peuvent aider. En expliquant aux héros que les holocrons qui sont comme celui là verrouillé sur l’esprit de leur propriétaire, ont souvent une clé physique permettant de ne pas perdre le savoir une fois le propriétaire décédé. C’est on général un lieu ou un object fortement lié à ce dernier, duquel émane une Force caractéristique (comme le Monolithe).
    \end{rebelist}
\end{paperbox}

\paragraph{Holocron ouvre toi}
Placer dans l’emplacement adéquat sur le Monolithe, l’Holocron s’ouvre et libère ses secrets. C’est un mélange de visuel et de perseption mentale, les héros sensible à la Force comprennent mieux ceux qu’ils voient, les autres font un jet de \textit{Vigueur+2} pour ne pas s’évanouir.

On voit (et on ressent) \nameref{sec:karness-muur} en train de concevoir l’artéfact, étape par étape. Un jet de \textit{Maîtrise de la Force} pour les héros possédant l'Atout \textit{Jedi} ou \textit{Sith} permet a ce dernier de comprendre ce que Muur est en train de faire.

Néanmoins, tous ceux qui ne se sont pas évanoui remarquent et ressentent quelque chose en assistant à la fabrication du Talisman. Ce dernier n’est pas parfait, Muur n’a fait qu’un seul essai et il a été hésitant leur de certaines étapes. Le Talisman a donc de grandes chances de posséder des micro-fissures, suffisantes même pour le détruire à condition de posséder un pouvoir immense, au dela même de celui d’un simple Jedi. Mais il est probable qu’un impact avec un projectile ou un sabre, chargé de Force, étire les fissures suffisament pour que l’hôte, au prix d’un effort considérable, parvienne à reprendre le dessus et se libère du Talisman. 

Dans ce cas, pendant un court laps de temps, il serait possible de détacher l’artéfact et de le jeter dans l’oubliette. L’ancien hôte demeurerait toutefois incapable de se défendre et vidé de ses forces pendant un moment.

\subsection{La bataille finale}
Avec leur plan en tête vos héros partent donc pour la bataille finale, sur la lune de Lothal/Yavin (selon leur camp). Ils ont les coordonnées approximative de la zone de test où se trouve l’oubliette. Dés qu’ils survolent la zone, un Force inconnue attire le Nimbus au sol et l'oblige à se poser. Les héros se retrouvent dans une sorte de cratère, tout autour d’eux, des centaines de Rakgouls sont rassemblés. Au loin, à pas loin d’1~km se trouve Céleste Moorne qui les regarde. Manifestement, l’oubliette a été ouverte !

Avant qu’ils n’ai eu le temps d’y réfléchir, une première vague de Rakghouls (prévoir 10 / 12) se dirige vers eux à grande vitesse. Ils peuvent se servir des canons du vaisseau pour éliminer la première vague. Une fois la première vague éliminé, il ne se passe rien tant qu’ils ne sortent pas du vaisseau. Ils ne peuvent pas faire redécoller le vaisseau. S’il ne parviennent pas a éliminer la première vague en 50 tours, les Rakgouls pénaitrent dans le vaisseau, il faudra les finir à la main.

Une fois sorti du vaisseau, la deuxième vague de Rakghouls (prévoir 1.5 Rakghoul par héros) s’élance vers eux tandis que Céleste se contente d’observer de loin. Laissez les héros engager le combat, s’ils ne s’en sortent pas ou quand il ne reste qu’1 ou 2 enemi, lancer une nouvelle vague, massive cette fois avec des \nameref{sec:rakghoul-amblyope} en prime, ça arrive de tout les cotés. Il faut que vos joueurs se sentent perdu. 

\paragraph{Résistance}
Puis quand ils ont bien paniqué, l’Uhumele sort de la couche nuageuse et commence à canarder dans tout les sens sur les Rakghouls. Et là, c’est Dass Jennir qui saute du vaisseau et qui vient se placer au coté des héros, suivit de tout l'équipage de \nameref{sec:schurk-heren}. 
\begin{quotebox}
    \nameref{sec:dass-jennir}: On s’occupe de vous ouvrir la voie, faite ce que vous devez faire !
\end{quotebox}

\paragraph{Empire}
Puis quand ils ont bien paniqué, un Croiseur légé de l’Empire sort de la couche nuageuse et commence à canarder dans tout les sens sur les Rakghouls. Et là, c’est \nameref{sec:garan-keggle} qui saute du vaisseau et qui vient se placer au coté des héros, suivit d’un contingeant de \nameref{sec:storm-trooper}. 
\begin{quotebox}
    \nameref{sec:garan-keggle}: On s’occupe de vous ouvrir la voie, faite ce que vous devez faire !
\end{quotebox}

\paragraph{Commun}
La bataille fait rage, ça part dans tout les sens, de temps à autre faite un lancer de dés, si plus de 4, les héros se trouvent face à face à autant de Rakghouls que le dés dépassent 4 (Ex: 6 = 2 Rakghouls). Ils sont alors obligé de les affronter. 

Après moulte batailles, les héros se retrouvent face à Céleste Morne, possédé par l’esprit de Muur qui s’adresse aux héros (intimidation). Au lieu d'une phrase, Karness semble avoir du mal a parler, il semble se battre contre un démon intérieur ... 
\begin{quotebox}
    \nameref{fig:karness-muur}: Fuyez vous ne pourrez pas le maitriser ... 
\end{quotebox}

C’est alors que commence la phase de combat. A chaque tour, Céleste fait un jet \textit{d’\^Ame}, si le jet est réussit, Karness attaque, si le jet est raté, Céleste parvient à le retenir en cas d’échec critique, Karness est secoué.

Les héros doivent charger une arme (Sabre Laser ou Phaser) avec la Force. S’il n’y a personne de sensible à la Force dans le groupe, faite venir \nameref{sec:dass-jennir} ou \nameref{sec:garan-keggle} avec eux jusqu’à Céleste afin de charger l’arme.

Peu importe qui charge l’arme, il ne pourra rien faire d’autre pendant son tour. Pour charger l’arme il doit réussir 2 jet de \textit{Maîtrise de la Force}.

Ensuite il faudra que quelqu’un (l’un des héros PJ) utilise l’arme et réussisse l’action. S’il touche c’est gagné, s’il rate, il faut recommencer. En attendant, les héros sont confronté à Karness ou à des petits groupe de Rakghouls. Si les héros frappe Céleste trop facilement, faite leur frapper un deuxième coup pour que ça fonctionne.

\subsection{Epilogue}
Le Talisman est touché, Céleste se met à hurler et une vague de Force Lumineuse part de son corps et s’étend sur l’ensemble du cratère. Les Rakghouls touchés par cette vague perdent toute cohésion et combativité. Le Talisman se détache alors du poignet de Céleste et tombe à terre en même temps que Céleste s’écroule au sol, en quelques seconde son corps se met à veillir jusqu’à devenir poussière. Le Talisman la maintenait en vie depuis plus de 1000~ans.

L’oubliette se trouve à 200~m derrière Céleste, c’est à vos héros de voir ce qu’ils font. Mais s’ils ne font rien dans les 2~mn, le Talisman va s’accrocher à \nameref{sec:dass-jennir} ou \nameref{sec:garan-keggle} et dans ce cas tout est perdu. Normalement ils devraient balancer le Talisman dans l’oubliette et refermer.

Bon là c’est la méga-happy-end à vous de voir.

\subsubsection{Progression}
Les héros reçoivent 4~XP pour ce scénario. Ils reçoivent aussi un Atout \textit{Contact} en la personne de \nameref{sec:dass-jennir} ou \nameref{sec:garan-keggle} qui les appréciera pour les qualités dont ils ont fait preuve dans cette mission.