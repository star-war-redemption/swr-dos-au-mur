\section{Dos au Muur}
On attaque ici le cinquième et dernier scénario de la campagne.

\subsection{Résumé des épisodes précédents}
Le camp oposé (l’empire pour les gentils et l’alliance si vos héros sont avec Dark Vador) a récupéré l’\nameref{sec:oubliette-de-dreypa} au nez et à la barbe de vos héros qui sont donc bien dégouttés. Possiblement, vos héros ont trouvé l'Holocron de Muur dans le laboratoire de \nameref{sec:pulsipher} mais pas forcément. Enfin, vos héros savent ce que contient l’oubliette.

Vos héros vont donc errer un moment dans l’Univers à la recherche d’informations qui pourraient les mener à l’oubliette perdue. Selon que vos joueurs se montrent imaginatif ou pas sur la recherche, abrégez ou non la séquence par un appel de leur hiérarchie qui aura une info à leur faire part. Si vos joueurs se montrent imaginatif faites leur trouver la dite information par eux mêmes.

\subsection{Sur la piste}
\paragraph{Empire}
Si vos héros sont à la solde de l’Empire, l’information qui finie par leur parvenir est que des espions infiltrés ont découvert au péril de leur vie que les rebels ont améné sur une base reculée de l’alliance, une cargaison très spéciale et que depuis seul le personnel fortement accrédité est autorisé à circuler dans la zone de stockage de la cargaison.

La base rebelle se trouve sur la lune IV de Yavin. Les rebels sont établi dans ce système depuis plusieurs mois. La quatrième lune de Yavin est une lune forestière avec une végétation dense offrant un bon camouflage aux batiments installés sous la canopée. La lune possède plusieurs installations distante les unes des autres de plusieurs centaines de kilomètres. La fameuse cargaison est entreposée et sans doute étudiée dans l’installation la plus au Nord et la plus éloignée de toute les autres.

\paragraph{Alliance}
Si vos héros se sont enrolés dans la résistance, c’est un peu la même histoire. L’information qui leur parvient est que l’Empire à fait porter sur une lune de Lothal une cargaison et que depuis le batiment où est entreposée la cargaison est verrouillé. 

Sur l’une des lunes de Lothal, l’Empire possède une base militaire. La lune est de type désertique. Pas de végétation si d’eau, seulement une atmosphère un peu raréfiée. La cargaison est entreposée 100~km au Nord de la base dans un zone qui depuis est fermée à tous les soldats non autorisés.

\paragraph{Commun}
Dans tous les cas, les héros sont prévenu que le camp adverse s’emploit à ouvrir l’Oubliette mais que pour l’instant il n’y est pas parvenu.

Donc là normalement vos héros sentent (ou non) le piège, il leur faut donc un plan. Ils savent que dans l’oubliette se trouve \nameref{sec:celeste-morne}, une puissante Jedi et qui plus est, possédée par le Talisman de Muur, un non moins puissant Sith ! Le combat, si combat il y a, s'annonce un peu difficile. A vous de leur faire comprendre s’ils ne le ressentent pas spontanément, qu’y allez sans un plan pour libérer Céleste du Talisman n’est rien de plus qu’un suicide collectif.

\subsection{Man with a plan}
Le camp oposé prépare un piège et fait passer une information comme quoi l’oubliette se trouve sur une lune je ne sais pas où et qu'ils essayent de l’ouvrir. L’idée du piège et d’ouvrir l’oubliette et avec plein de gens autour pour transformer tout le monde en Rakgoul qui se chargeront de zigouiller tout ceux qui seront venu chercher l’oubliette.
A voir pour l’alliance, c'est pas très éthique comme façon de faire.

Les héros ont donc vent de la chose par leur hiérarchie. Mais avant de se jeter tête baissé dans le piège ils devront se renseigner sur comment ouvrir l'oubliette et comment détacher le Talisman de son porteur.

pour cela, soit ils ont avec eux l'holocron, l'idée est alors d'arrivé à l'ouvrir. Sinon, ils devront se renseigner à l'ancienne dans les archives d'un temple sith. Pour les héros coté obscur, croiser Drak Vador serait pas mal.

Une fois qu'ils savent comment libérer Céleste Moorne de l'emprise du Talisman ils peuvent partir à la rencontre. Le mieux est de faire genre "on va chercher ou se trouve l'oubliette, pendant ce temps chercher quoi en faire".


Une fois sur la lune, c'est le boss de fin de campagne !
Dés que le vaisseau survole la zone, il est attiré par la Force au sol. Une fois posé, les héros se retrouvent dans une sorte de cratère et des dixaines de Rakgoul autour d'eux. Au fond se trouve Céleste Moorne. Le combat s'engage contre la première vague de rackoul (doser le nombre en fonction des héros). Une fois la première vague à terre, la deuxième vague s'avance, Céleste au fond ne bouge pas.
Mais avant que la deuxième vague n'arrive au contact, l'Uhumele apparait de derrière les nuages et commence à déglinguer la vague de rackoul. Et là, c'est Dass Jenir qui saute du vaisseau et qui vient se placer au coté des héros, suivit de tout l'équipage de Schurk Heren. "On s'occupe de vous ouvrir la voie, faite ce que vous devez faire"

La bataille fait rage, les héros se retrouvent face à Céleste Morne, possédé par l'esprit de Muur qui s'adresse aux héros (intimidation). Au lieu d'une phrase, Karness semble avoir du mal a parler, il semble se battre contre une démon intérieur ... "Fuyez vous ne pourrez pas le maitriser ..."

Commence la phase de combat. A chaque tour, Céleste fait un jet d'Ame, si le jet est réussit, Karness attaque, si le jet est raté, Céleste parvient à le retenir en cas d'échec critique, Karness est secoué.

Quelque soit la façon dont les héros vont libérer Céleste, ça sera une action qui devra se faire en 3 réussite. Chaque réussite augmente les chance de Céleste de maitriser Karness (malus de -2 à chaque fois ?). 