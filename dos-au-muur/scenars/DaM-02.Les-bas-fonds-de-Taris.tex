\section{Les bas fonds de Taris}

Taris est une planète urbaine très polluée de la Bordure Extérieure. Divisée en trois grandes parties en fonction des étages, la Ville Haute, la ville médiane et la ville basse aussi appelée "bas fonds", elle abrite une population gigantesque. Grand pôle économique dans la galaxie, notamment grâce à l’implantation des industries Lhosan, elle joignit la République Galactique peu avant les Guerres Mandaloriennes. 

\subsection{Résumé des épisodes précédents}
Nos héros se retrouvent donc dans un Cargo léger YZ-775 sans canons fonctionnel et un message entrant d’une inconnue qui demande à parler à Tinon.

Ils ont aussi apris qu’Industrial Automaton sous couvert d’expérimentations de son nouveau modèle, fait des recherches sur un très ancien artefact Sith en provenance de \textbf{Taris}. De plus, s’ils ont réussi à récupérer le droïde, Vyna Anen les attend sur l’avant-poste commercial de \textbf{l’anneau de Kafrene} au Starlord Café. Enfin, ils savent aussi que le vaisseau se dirigeait vers \textbf{Gaulus} pour rejoindre un laboratoire de l’empire où il devait finir des recherches sur l’artéfact.

Plusieurs directions s’offrent alors à nos héros.

\subsection{Les bon, les brutes ou les truants}
\begin{quotebox}
    Tinon \emph{(prononcer Taïnon)} c’est toi ? Que se passe t’il ou va tu ?
\end{quotebox}
Le visage d’une jeune femme apparait sur l’holocomm. Les héros ont le choix de répondre ou non.

Ce point de l’aventure est crutial pour nos héros car ils vont choisir (en connaissance de cause ou non) l’orientation de leurs personnages. En effet plusieurs cas de figures se présentent :

\begin{description}
    \item[\nameref{sec:les-rebels}] Répondre au message et suivre les consignes des Rebels (page~\pageref{sec:les-rebels}).
    \item[\nameref{sec:retour-du-droide}] Ramener le droïde à Vyna Anen (page~\pageref{sec:retour-du-droide}).
    \item[\nameref{sec:refus-d-obtemperer}] Partir pour Taris directement (page~\pageref{sec:refus-d-obtemperer}).
    \item[\nameref{sec:l-empire}] Prendre la direction de Gaulus (page~\pageref{sec:l-empire}).
\end{description}

Tous ses choix amèneront de toute façon les héros sur Taris mais le chemin sera différent et par là même leur orientation aussi. Je vais tacher de décrire les choix possibles sachant que rien n’est écrit dans le marbre et vous pouvez très bien les forcer suivre une route en particulier.


\subsection{Les rebels} \label{sec:les-rebels}

\begin{quotebox}
    Tinon \emph{(prononcer Taïnon)} c’est toi ? Que se passe t’il ou va tu ?
\end{quotebox}

Le visage d’une jeune femme apparait sur l’holocomm et les héros choisissent de répondre.

\begin{quotebox}
    Mais qui êtes vous et où et Tinon ?
\end{quotebox}
Les joueurs choisissent de raconter leur histoire, ou de mentir, à voir ce qu'ils préfèrent. Lindi leur demande de la rejoindre et leur donne les coordonnées d'une cache de la Rébellion. Si les joueurs ont menti, il sont accueilli avec suspicion, menacé et tout, sinon il sont accueillit normalement avec juste un peu de méfiance. \textbf{Lindi Dangon} les invite à la suivre et les interroge jusqu'a ce qu'ils disent la vérité (On part du principe que s'ils persistent à mentir, on bifurque vers l'option \textbf{\nameref{sec:refus-d-obtemperer}}).

Lindi leur explique alors que la résistance à eu vent des recherches mené par l'Industrial Automaton ainsi que des problèmes sur le Pelican. \textbf{Tinon Dystra} avait été envoyé pour tenté de s'infiltrer sur le vaisseau en profitant du désordre régnant à bord. Mais qu'il n'avait plus donné de nouvelles depuis son arrimage au Pelican.

Un jet de \textbf{Perception} leur apprendra que Lindi et Tinon étaient amant.

\subsection{Retour du droïde} \label{sec:retour-du-droide}

\subsection{Refus d’obtempérer} \label{sec:refus-d-obtemperer}

\subsection{L’empire} \label{sec:l-empire}

-----


ici les héros sont dans un navette et s’échape du Pelican qui fonce droit sur une étoile.

Une communication entrante les interpelle et leur demande des compte. Il s’agit de la résistance.

une fois l'appel terminé les héros disposent de trois choix :
 - Raméner le droide à Industrial Automaton
 - Rejoindre la résistance
 - Ou partir directement sur Gaulus au labo d'Industrial Automaton
 - voire un 4ieme de partir pour Taris

Bon l'idée c'est que dans tout les cas la prochaine étape est Taris et le temple Sith dans lequel ils trouverons un holocron qui leur dira que le Talisman est partie sur Jebble avec Céleste Morne et Pulcipher.
L'holocron parle de l'oubliette de Dreypa et de la rivalité entre Dreypa et Karness. Il ne dit pas ou se trouve l'oubliette ni le talisman. 
Pas loin se trouve les restes du campement de Pulcipher avec son journal qui explique qu'il a l'intention de partir pour Jebble à son labo pour y étudier l'artéfact. Le dernier message est coupé avant la fin mais il semble que Pulcipher et été interrompu par un duo de Jedi et ai du quitté Taris en précipitation.


L'épisode suivant se passe sur Jebble dans le labo du professeur Pulcipher. Où l'on apprend ce qu'il s'est passé pendant le voyage de retour et où l'on a des piste sur l'endroit ou se trouve l'oubliette. On entend notement parler de Céleste Morne.
Ici une idée est qu'au moment de quitté la planète pour l'étape suivante, les héros se retrouvent pris au piège par des troupes de l'empire qui on pris leur vaisseau en otage. Histoire de varier l'aventure. On peut même se mettre une petite baston spaciale.
Il faudra ensuite forcer les héros à retourner à leur QG (alliance rebelle ou empire) pour réparation et pour compte rendu.

Dans l'épisode suivant, le rebelles apprennent qu'on aurait vu le talisman sur une certaine lune de Jesaispasou et les soldats de l'empire apprenne qu'ils vont tendre un piège à l'alliance sur un lune de Jesaispasou.
Grossomerdo, Celeste se pointe et calme tout le monde, obligé de battre en retraite et de trouver un plan pour enfermer Celeste dans l'oubliette de Dreypa ou pour lui virer le Talisman avant d'enfermer ce dernier dans l'oubliette. Ou se trouve l'oubliette ? Quel est le plan ?