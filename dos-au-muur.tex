\documentclass{jdrp}

\bibliography{references} 

\newcommand*{\crg}{{\aurebesh\Large \$}} % Symbol for Galactic Credits

\hypersetup{  
  pdfinfo={  
    Title={SWR - Dos au Muur},
    Subject={Star Wars Redemption, campagne Dos au Muur, version 1.0.1},
    Author={Marthym},
    Keywords={star,wars,savage,worlds,jdra,jdrp,scenario,muur,talisman},  
    Copyright={Do What The Fuck You Want To Public License}
  }  
} 

\begin{document}

	\begin{titlepage}

	\begin{center}
		\hspace*{\vfill}
		\noindent\Huge\jedifont{Star Wars Redemption}\\ 
		\noindent\fontsize{50}{70}\jedifont{\$}
		\noindent\fontsize{50}{70}\jedifont{\#}\\
		\noindent\fontsize{50}{60}\jedifont{Dos au Muur}
		\hspace*{\vfill}
	\end{center}

	\noindent\makebox[\textwidth]{
		\includegraphics[width=1.01\paperwidth]{swr-class/_img/cover-bg.jpg}}
	\begin{tikzpicture}[overlay]
		\node[minimum width=180pt,minimum height=180pt, rotate=30] at (15,11){\includegraphics[width=180pt]{_img/talisman.png}};
	\end{tikzpicture}}
	\end{titlepage}

	\onecolumn
	\section{contexte de campagne}
	
	\begin{wrapfigure}{R}{180pt}
		\centering
		\includegraphics[width=180pt]{_img/talisman.png}
		\caption{\label{fig:talisman-de-muur}Talisman de Muur}
		\vspace{1\baselineskip}
		\includegraphics[width=180pt]{_img/pnjs/karness-muur.jpg}
		\caption{\label{fig:karness-muur}Karness Muur}
	\end{wrapfigure}
	
	Cette campagne est écrite initialement pour \citetitle{jdrp-starwars} mais un scénar reste un scénar et il est jouable dans n’importe quel univers de Star Wars.

	L’idée était de faire une campagne d’introduction avec des personnages partant de rien. Les joueurs commencent Novice et n’ont pas besoin d’historique complexe et élaboré (bien que cela ne soit pas interdit bien sûr). De cette façon, les personnages devraient être assez vite fait. Et elle est adaptable aussi bien avec des joueurs orienté Alliance Rebelle que Empire. Dans les deux cas, l’objectif sera le même mais les dessains changeront.

	La campagne se déroule dans les premières années de l’avènement de l’Empire, au MJ de voir s’il veut préciser.

	La trame de la campagne se base sur un très ancien artefact Sith, le \citetitle{talisman-de-muur}. Un artefact créé par Karness Muur, un Sith se servant de la Force pour prolonger sa vie. L’artefact contient l’âme de Muur, celui qui le porte est possédé par Karness et peut contrôler les Rakghoules.


	On trouve beaucoup d’informations sur cet artefact sur HoloNet et je me suis grandement inspiré de ces informations pour cette campagne en faisant vivre à mes héros les aventures de divers protagonistes ayant croisé le Talisman\ldots

    \subsubsection{Note sur le PDF}

    Ce PDF contient deux calques. La couleur de fond des pages peu être masqué pour une meilleure qualité d’impression.

	\subsubsection{Star Wars Redemption}
	\cite{jdrp-starwars}

	\subsubsection{Licence}
	\noindent DO WHAT THE FUCK YOU WANT TO PUBLIC LICENSE\\
    Version 2, December 2004
    \vspace{-2.5\baselineskip}
	\begin{flushright}
		\includegraphics[width=70pt]{swr-class/_img/wtfpl-badge.png}
	\end{flushright}

	\twocolumn

	\include{tex/scenars/DaM-01.Sauvetage-du-Pelican}
	\include{tex/scenars/DaM-02.Les-bas-fonds-de-Taris}
	\include{tex/scenars/DaM-03.Laboratoire-de-Pulsipher}
	\section{C’est dans la boite}


\subsection{Au rapport !}
Nos héros quittant Jebble son contacté par leur faction.

\subsubsection{Empire}
\begin{quotebox}
    \nameref{sec:garan-keggle}: Au rapport~! Comment avancent vos recherches sur l’artefact~?
\end{quotebox}
Les joueurs racontent\ldots
\begin{quotebox}
    \nameref{sec:garan-keggle}: La "Boite de Jebble"\ldots Ça ne me dit rien~! Mais allez voir \nameref{sec:fane-peturri} sur \textbf{Muunilinst}, c’est un historien de l’Empire, ami de notre Empereur, il pourra certainement vous aider. Je vais le prévenir de votre visite et je vous transfère ses coordonnées.
\end{quotebox}

Les héros se rendent donc sur \textbf{Muunilinst} (normalement) et rencontrent \nameref{sec:fane-peturri}. Ce dernier les invite chez lui, leur offre le thé et les reçoit bien.
\begin{quotebox}
    \nameref{sec:fane-peturri}: Garan m’a prévenu de votre arrivée, mais il ne m’a pas dit quel était le but de votre visite~? En quoi puis-je aider l’empire~?
\end{quotebox}
Les héros devraient donc lui parler de la fameuse boite de Jebble qui est en réalité \nameref{sec:oubliette-de-dreypa}.

\begin{quotebox}
    \nameref{sec:fane-peturri}: Il me semblait bien qu’il y avait un rapport entre l’oubliette et cette Boite~! Il y a quelque temps, suite à une réquisition d’\oe{uvres} d’art pour le compte de l’Empire, je suis tombé sur un enregistrement holo qui parlait de cette boite. D’après ce que disait l’enregistrement, mais il date maintenant de plusieurs mois, la \textbf{Boite de Jebble} se trouve sur le \nameref{sec:uhumele} un cargo qui verse dans le commerce et la contrebande de babioles diverses. Son capitaine, \nameref{sec:schurk-heren} est un Yarkora qui se méfie de tout le monde en général et en particulier de l’Empire. Dites-lui que vous êtes des amis de "\textbf{\nameref{sec:uhumele-janks}}" mais il n’est pas idiot donc prenez ce dossier sur Janks et étudiez le avant de vous faire passer pour ses amis.

    Le dernier port d’attache que je lui connais est \textbf{Pizkoss}, il est certainement en train de dépenser ses crédits chez \nameref{sec:queen-jool}.
\end{quotebox}

\subsubsection{Rebelion}
\begin{quotebox}
    \nameref{sec:lindi-dangon}: Bonjour, je viens aux nouvelles, comment se passe la recherche de l’artéfact~? Vous avez besoin de quelque chose~?
\end{quotebox}

Les joueurs racontent\ldots

\begin{quotebox}
    \nameref{sec:lindi-dangon}: La \textbf{Boite de Jebble}, ça me dit quelque chose, attendez une seconde\ldots 

    \textit{Lindi disparaît de l’écran un instant puis revient}

    \nameref{sec:lindi-dangon}: Effectivement, un Jedi en parle dans un de ses rapports. \nameref{sec:dass-jennir}, il n’est pas très précis dans son rapport, mais vous devriez aller le voir. Il est en retraite sur \textbf{Muunilinst}. Je vous fais suivre les coordonnées où vous le trouverez, soyez respectueux et diplomates~! Rappelez-vous que c’est un Jedi~!
\end{quotebox}

Les héros se rendent donc sur \textbf{Muunilinst} (normalement) pour rencontrer \nameref{sec:dass-jennir}. Dass Jennir se montre tout d’abord froid et distant
\begin{quotebox}
    \nameref{sec:dass-jennir}: C’est pour quoi~? Si c’est encore pour réparer votre moissonneuse, revenez plus tard, je suis occupé là.
\end{quotebox}

Aux héros de se montrer diplomate pour l’amadouer~! Quand c’est fait. Ils lui parlent de la Boite de Jebble. En entendant ce nom, le visage de Dass Jennir marque un sentiment de souffrance et de regret. C’est manifestement un souvenir douloureux pour lui.
\begin{quotebox}
    \nameref{sec:dass-jennir}: Oui, je me souviens de ça~! Je ne l’ai pas vu personnellement, mais elle faisait partie de l’inventaire de l’\nameref{sec:uhumele} quand je suis passé à son bord pour une mission. L’équipage du vaisseau avait à son bord cette chose, aux dernières nouvelles ils partaient sur \textbf{Pizkoss} pour tenter de trouver un acheteur pour cette cargaison.

    Vous devez savoir que déjà à l’époque, l’Empire recherchait ardemment cette cargaison, c’est d’ailleurs ce qui explique qu’il ne l’ai toujours pas revendue. 

    Le capitaine du cargo s’appelle \nameref{sec:schurk-heren} je vais vous dire où le trouver sur \textbf{Pizkoss} mais je ne viendrais pas avec vous, j’en ai fini avec tout ça. Habituellement, \nameref{sec:schurk-heren} dépense ses crédits chez \nameref{sec:queen-jool} quand il fait escale à \textbf{Pizkoss}.
\end{quotebox}

\newpage
\subsection{Pizkoss’ Paradise}
Pizkoss est un Monde du Noyau. En tant que tel, le commerce y est très prospère et on y retrouve à peu près toutes les races possibles. Toutes sortes de transaction et ont lieu des plus légales aux plus douteuses. En tant que planète du noyau elle est surveillée par l’Empire.

\nameref{sec:queen-jool} est la propriétaire d’une cantina libertine sur \textbf{Pizkoss}, le \textbf{Paradise}. C’est un club "select" où l’on ne rentre que sur invitation. Il ne faut pas chercher à rentrer de force au risque de se faire expulser violemment et d’attirer par la même l’attention de l’Empire. Diplomatie et astuce (déguisement, soudoiement, propositions indécentes \ldots) sont de rigueur. Une fois à l’intérieur, attention de ne pas faire de vagues~! Au moindre problème, vigiles et \nameref{sec:storm-trooper}s débarquent et enferment tout le monde.\\

\noindent\includegraphics[width=\linewidth]{_img/places/paradise-club.png}\\

\'A l’intérieur la salle se présente comme un stripbar classique. Un bar sur la droite, une estrade au fond avec un podium qui avance sur la moitié de la pièce. Des îlots plus privés tout au tour et des salons privatif sur la gauche. 

Les héros ne savent pas à quoi ressemble \nameref{sec:schurk-heren} ils n’ont que son nom, à eux de le retrouver. En vrai il se trouve dans l’un des salons privés où il profite d’une danse avec \textbf{Na’tuna}, une charmante Twi’lek en tenue légère.

Quand les héros rentrent dans le salon, Schurk est un peu surpris, Na’Tuna reste imperturbable, professionnalisme avant tout.

\begin{quotebox}
    \nameref{sec:schurk-heren}: Ben faut pas vous gêner~! Décidément tout se perd, les bonnes manières y compris~! Je ne sais pas ce que vous me voulez mais vous ne pensez pas que ça peut attendre que la dame est terminée~?
\end{quotebox}

Gérez Schurk en fonction du comportement de vos héros, quand ils se mettent à discuter~:

\begin{quotebox}
    \nameref{sec:schurk-heren}: Bon, maintenant que l’on est entre nous, en quoi puis-je vous aider ?\\
    \textit{\textbf{Joueurs}}: \ldots\\
    \nameref{sec:schurk-heren}: Humm, alors comme ça cet inestimable et antique coffre vous intéresse ? C’est fâcheux, figurez-vous que je viens tout juste de le promettre à un client fortuné. Il serait vraiment très déçu si je lui faisais faux bon.\\
    \textit{\textbf{Joueurs}}: \ldots\\
    \nameref{sec:schurk-heren}: Cependant vous m’êtes sympathique~! Il se pourrait que je vous dise où se trouve la cargaison. Si vous êtes assez rapide, vous pourrez la récupérer avant mon autre client. Je n’ai pas encore rencontré mon client pour la transaction, ça vous laisse le temps de me rendre un petit service.\\
    \textit{\textbf{Joueurs}}: \ldots \\
    \nameref{sec:schurk-heren}: Mais vous êtes des amis de \textit{[\textbf{Dass Jennir}|\textbf{Janks}]}, je ne me vois pas vous refuser ça~! Il se trouve que je n’ai pas une confiance aveugle en ce client. Il est connu pour être un criminel et je m’attends à ce qu’il essaye de me doubler. Dans ces conditions, je vous propose d’assister à la transaction de loin. Si ça tourne mal, on se débarrasse de lui et je vous fais la cargaison à prix d’ami.

    Si tout se passe bien, vous connaîtrez l’emplacement de la cargaison et n’aurez qu’a vous y rendre avant eux. C’est ce que je peux vous proposer de mieux.
\end{quotebox}

Peu importe le prix que vous négocierez avec les héros il n’y aura jamais besoin de le payer.
\begin{paperbox}{Songes d’Uhumele}
    \'A noter que si vos héros ont fait le scénar \citetitle{swr-songes-uhumele}, ils connaissent les membres de l’Uhumele ainsi que leur aventure avec \nameref{sec:dass-jennir} et ils ont le droit d’en jouer. Ils y sont même encouragé.
\end{paperbox}
\subsection{La transaction}

La transaction à lieu dans un entrepôt à l’extérieur de la ville. L’Uhumele est stationné à quelques pas, non loin de l’entrepot, nos héros restent dans le cargo alors que le capitaine et son équipage partent avec un fausse cargaison pour traiter avec le client. Les héros distinguent ce qu’il se passe de loin.

Tout semble bien se passer mais brusquement le client, un \textbf{Ishi Tib} apparemment, sort une arme. Schurk-Heren tente d’en faire autant, mais elle lui est oté des mains par un tir de blaster venant de sa droite en hauteur. Depuis le vaisseau, les héros voient sortir une douzaine d’hommes armé de \textit{Fusil Blaster} qui encercle l’équipage.

Les héros peuvent ici choisir d’entrer dans le combat ou non. Le nombre d’adversaire à combattre est important et doit les obliger à mettre au point une stratégie de repli permettant de sauver au moins l’un des membres d’équipage, vu qu’ils ne savent toujours pas où se trouve la cargaison. Si les héros entrent dans le combat, laissez passer quelques tours de combat, quand les héros commencent à souffrir un peu, ou s’ils s’enfuient, passez à la suite.\\

Donc au bout d’un moment, un vaisseau apparaît au-dessus de l’entrepôt et fait exploser le toit (Si tous les joueurs sont dans l’entrepôt, ils ne le voient pas arriver). Le vaisseau prend tout le monde pour cible, peu importe le camp, pendant qu’un rayon tracteur fait main basse sur la fausse cargaison.
Les hommes du \textbf{Ishi Tib} sont encore nombreux mais sont distrait par le vaisseau qui vole leur cargaison.

Le capitaine de l’Uhumele prend un tir dans l’épaule qui le blesse gravement. Deux autres de ses équipiers se font blesser à leur tour. Voyant la situation mal tourner, les deux blessés demandent aux héros et au reste de l’équipage de prendre leur capitaine et de partir se mettre à l’abri pendant qu’ils couvrent leur fuite. Les héros parviennent à atteindre le vaisseau mais les hommes de l’\textbf{Ishi Tib} se ressaisissent et les héros voient les deux hommes d’équipage capturés par le \textbf{Ishi Tib}.

Une fois à bord, \textbf{Schurk-Heren} commande à l’Uhumele de dégager et de mettre le cap sur \textbf{Pizkoss}. Une fois en sécurité, 
\begin{quotebox}
    \nameref{sec:schurk-heren}: Ce vaisseau, je le connais, c’est celui du bras droit du \textbf{Ishi Tib}. Il aura eu les yeux plus gros que le ventre et aura voulu la cargaison pour lui seul.\\
    \nameref{sec:schurk-heren}: Je dois partir secourir mes hommes. S’ils sont interrogés, ils finiront par avouer que la cargaison était un leurre et il ne faudra pas longtemps avant qu’ils soient forcés de donner l’emplacement de la vraie. Il faut la récupérer au plus vite. Vous irez chercher la cargaison pendant que nous irons secourir le reste de mon équipage.\\
    \nameref{sec:schurk-heren}: La cargaison se trouve dans un champ d’astéroïdes, je vous transfère les coordonnées.
\end{quotebox}

Les joueurs reprennent le Nimbus sur \textbf{Pizkoss} et partent vers les coordonnées données par \textbf{Schurk-Heren}.

\subsection{Le champ d’astéroïde}
\noindent\includegraphics[width=\linewidth]{_img/places/asteroid-field.png}\\

Donc nos héros s’en vont chercher ce pour quoi ils sont venus, l’\nameref{sec:oubliette-de-dreypa} alias "la boite de Jebble". Mise à l’abri dans un champ d’astéroïdes par \nameref{sec:schurk-heren}.

Le champ d’astéroïde oblige les vaisseaux à sortir d’hyper-espace au large des coordonnées indiquées par Schurk. Dès la sortie, tous les voyants du \nameref{sec:nimbus} passent au rouge. En effet, ils ne sont pas seuls sur les lieux, une \nameref{sec:empire-corvette} est présente sur la zone. Quelques instants après leur arrivée, 4 \nameref{sec:tie-fighter} sont largués de la corvette et se dirigent vers le Nimbus pendant que la corvette s’engouffre dans le champ d’astéroïdes.

Un affrontement spatial s’engage entre le \nameref{sec:nimbus} et les 4 \nameref{sec:tie-fighter}. \textit{Les spécificités offensives et défensives des différents vaisseaux sont disponibles dans le \nameref{sec:bestiaire} et dans la section \nameref{sec:nimbus}}.\\

Une fois les 4 chasseurs explosés, les héros se précipitent (en principe) sur la zone où se trouve l’oubliette. Un cargo léger est en train de récupérer la cargaison de l’Uhumele quand le Nimbus arrive à portée de la corvette.

Les héros peuvent engager le combat, mais entre les dégâts encaissés contre les chasseurs et le fait que la corvette est mieux armée, le combat n’est pas égal. Ils peuvent prendre le cargo pour cible mais s’ils parviennent à l’exploser, la corvette utilisera un rayon tracteur pour récupérer la boite.

Au final, la corvette prendra la fuite sans demander son reste et les héros n’ont que leurs yeux pour pleurer.

\subsection{\'Epilogue}
C’est, pour une fois, un scénar qui ne se termine pas bien. On peut pas toujours gagner !

Pour l’XP, entre \textbf{2 et 3 XP} selon le jeu et la cohérence des joueurs.

\clearpage
\subsection{l’Uhumele}\label{sec:uhumele}
\noindent\includegraphics[width=\textwidth]{_img/uhumele-pano.png}
\\

L’Uhumele est un vaisseau cargo de classe inconnue, actif notamment pendant et après la Guerre des Clones. Son capitaine est le Yarkora Schurk-Heren. On ignore quand et dans quelles conditions Schurk-Heren devint capitaine de l’Uhumele et comment son équipage le rejoignit. Ce qui est sûr, c’est qu’il a de bonnes raisons de détester la République et de craindre l’Empire qui lui succède.

L’Uhumele est avant tout un vaisseau de contrebande, comme un bon nombre de cargos en apparence en règle. De ce fait, il est doté d’un armement susceptible de pouvoir le sortir des situations délicates dans lesquelles il se fourre. L’une de ces armes est un bras rétractable situé sous le ventre de l’appareil, au bout duquel se trouve une petite tourelle blaster, ressemblant à celle des Canonnières clones. Bien que cette tourelle permette d’avoir un très vaste angle de tir, elle rend la situation du tireur précaire car très exposé. 

\hspace{5\baselineskip}
\noindent\includegraphics[width=0.7\textwidth]{_img/uhumele.png}

\newpage
\vspace*{11\baselineskip}
\subsubsection{Janks}\label{sec:uhumele-janks}
On sait très peu de choses sur le Phindien Janks, hormis le fait qu’il faisait partie de l’équipage de l’Uhumele en -19, au crépuscule de la Guerre des Clones. Son poste était celui d’ingénieur assistant de Ratty, le Tintinna ingénieur en chef. Janks n’était pas vraiment doué pour les plans complexes, c’est pourquoi il se contentait de suivre son équipage sans maugréer.

Pendant que le Capitaine Heren allait à la pêche aux renseignements, Janks se chargea d’acheter des provisions pour l’équipage. Alors qu’il retournait au vaisseau, il tomba nez à nez avec une patrouille impériale. Janks ne fut pas aussi rapide, d’autant plus qu’il avait été pris totalement par surprise. L’Empire se saisit de lui et le plaça en détention\ldots

Janks fut placé dans une cellule à bord d’un destroyer stellaire impérial et torturé pendant qu’on l’interrogerait sur l’équipage du Uhumele et sa cargaison.

	\section{Dos au Muur}
On attaque ici le cinquième et dernier chapitre de la saga.

\subsection{Résumé des épisodes précédents}
Le camp oposé (l’empire pour les gentils et l’alliance si vos héros sont avec Dark Vador) a récupéré l’\nameref{sec:oubliette-de-dreypa} au nez et à la barbe de vos héros qui sont donc bien dégouttés. Possiblement, vos héros ont trouvé l'Holocron de Muur dans le laboratoire de \nameref{sec:pulsipher} mais pas forcément. Enfin, vos héros savent ce que contient l’oubliette.

Vos héros vont donc errer un moment dans l’Univers à la recherche d’informations qui pourraient les mener à l’oubliette perdue. Selon que vos joueurs se montrent imaginatif ou pas sur la recherche, abrégez ou non la séquence par un appel de leur hiérarchie qui aura une info à leur faire part. Si vos joueurs se montrent imaginatif faites leur trouver la dite information par eux mêmes.

\subsection{Sur la piste}
\paragraph{Empire}
Si vos héros sont à la solde de l’Empire, l’information qui finie par leur parvenir est que des espions infiltrés ont découvert au péril de leur vie que les rebels ont améné sur une base reculée de l’alliance, une cargaison très spéciale et que depuis seul le personnel fortement accrédité est autorisé à circuler dans la zone de stockage de la cargaison.

La base rebelle se trouve sur la lune IV de Yavin. Les rebels sont établi dans ce système depuis plusieurs mois. La quatrième lune de Yavin est une lune forestière avec une végétation dense offrant un bon camouflage aux batiments installés sous la canopée. La lune possède plusieurs installations distante les unes des autres de plusieurs centaines de kilomètres. La fameuse cargaison est entreposée et sans doute étudiée dans l’installation la plus au Nord et la plus éloignée de toute les autres.

\paragraph{Alliance}
Si vos héros se sont enrolés dans la résistance, c’est un peu la même histoire. L’information qui leur parvient est que l’Empire à fait porter sur une lune de Lothal une cargaison et que depuis le batiment où est entreposée la cargaison est verrouillé. 

Sur l’une des lunes de Lothal, l’Empire possède une base militaire. La lune est de type désertique. Pas de végétation si d’eau, seulement une atmosphère un peu raréfiée. La cargaison est entreposée 100~km au Nord de la base dans un zone qui depuis est fermée à tous les soldats non autorisés.

\paragraph{Commun}
Dans tous les cas, les héros sont prévenu que le camp adverse s’emploit à ouvrir l’Oubliette mais que pour l’instant il n’y est pas parvenu.

Donc là normalement vos héros sentent (ou non) le piège, il leur faut donc un plan. Ils savent que dans l’oubliette se trouve \nameref{sec:celeste-morne}, une puissante Jedi et, qui plus est, possédée par le Talisman de Muur, un non moins puissant Sith ! Le combat, si combat il y a, s’annonce un peu difficile. A vous de leur faire comprendre, s’ils ne le ressentent pas spontanément, qu’y allez sans un plan pour libérer Céleste du Talisman n’est rien de plus qu’un suicide collectif.

\subsection{Man with a plan}
C’est là qu’intervient l’Holocron de Muur.

S’ils n’ont pas trouvé, et bien c’est qu’ils ont mal joué ! La meilleure solution est encore de leur faire comprendre qu’il leur manque une pièce du puzzle et qu’ils doivent retourner sur \nameref{sec:jebble} chercher l’Holocron. Par exemple on leur rappelant que le datapad de \nameref{sec:pulcipher} faisait référence à un object triangulaire.

S’ils ont l’Holocron avec eux, il est temps de leur faire comprendre que c’est un élément important de l’histoire et qu’il va falloir trouver comment l’ouvrir. Pour l’ouvrir ils vont devoir retourner sur \nameref{sec:taris} dans les ruines de l’ancien temple Sith. Sur le Monolithe en y regardant d’un peu plus prêt on trouvera un emplacement triangulaire où l’Holocron entre à la perfection. Cette fois, on leur facilite le passage, pas de gros boss ni de complication particulière. C’est comme dans les jeux une fois que la zone est visitée elle est sécure. Et puis l’objectif de la mission n’est pas de refaire le scénario 2.

\newpage
\begin{paperbox}{Comment les amener sur Taris ?}
Quelques idées sur comment les amenà à retourner sur Taris pour ouvrir l’Holocron. Car ce dernier étant verrouillé sur l’esprit de Muur, il n’est pas possible de l’ouvrir, même pour une apprenti Sith.

    \begin{rebelist}
        \item S’il y a un héros sensible à la Force parmis les joueurs, le plus simple et de lui donner une vision dans laquelle il voit le monolithe. Ou plus subtil, il voit les mêmes évènements que ceux décrit pas le Monolithe.
        \item Sinon, \nameref{sec:garan-keggle} ou \nameref{sec:dass-jennir} peuvent aider. En expliquant aux héros que les holocrons qui sont comme celui là verrouillé sur l’esprit de leur propriétaire, ont souvent une clé physique permettant de ne pas perdre le savoir une fois le propriétaire décédé. C’est on général un lieu ou un object fortement lié à ce dernier, duquel émane une Force caractéristique (comme le Monolithe).
    \end{rebelist}
\end{paperbox}

\paragraph{Holocron ouvre toi}
Placer dans l’emplacement adéquat sur le Monolithe, l’Holocron s’ouvre et libère ses secrets. C’est un mélange de visuel et de perseption mentale, les héros sensible à la Force comprennent mieux ceux qu’ils voient, les autres font un jet de \textit{Vigueur+2} pour ne pas s’évanouir.

On voit (et on ressent) \nameref{sec:karness-muur} en train de concevoir l’artéfact, étape par étape. Un jet de \textit{Maîtrise de la Force} pour les héros possédant l'Atout \textit{Jedi} ou \textit{Sith} permet a ce dernier de comprendre ce que Muur est en train de faire.

Néanmoins, tous ceux qui ne se sont pas évanoui remarquent et ressentent quelque chose en assistant à la fabrication du Talisman. Ce dernier n’est pas parfait, Muur n’a fait qu’un seul essai et il a été hésitant leur de certaines étapes. Le Talisman a donc de grandes chances de posséder des micro-fissures, suffisantes même pour le détruire à condition de posséder un pouvoir immense, au dela même de celui d’un simple Jedi. Mais il est probable qu’un impact avec un projectile ou un sabre, chargé de Force, étire les fissures suffisament pour que l’hôte, au prix d’un effort considérable, parvienne à reprendre le dessus et se libère du Talisman. 

Dans ce cas, pendant un court laps de temps, il serait possible de détacher l’artéfact et de le jeter dans l’oubliette. L’ancien hôte demeurerait toutefois incapable de se défendre et vidé de ses forces pendant un moment.

\subsection{La bataille finale}
Avec leur plan en tête vos héros partent donc pour la bataille finale, sur la lune de Lothal/Yavin (selon leur camp). Ils ont les coordonnées approximative de la zone de test où se trouve l’oubliette. Dés qu’ils survolent la zone, un Force inconnue attire le Nimbus au sol et l'oblige à se poser. Les héros se retrouvent dans une sorte de cratère, tout autour d’eux, des centaines de Rakgouls sont rassemblés. Au loin, à pas loin d’1~km se trouve Céleste Moorne qui les regarde. Manifestement, l’oubliette a été ouverte !

Avant qu’ils n’ai eu le temps d’y réfléchir, une première vague de Rakghouls (prévoir 10 / 12) se dirige vers eux à grande vitesse. Ils peuvent se servir des canons du vaisseau pour éliminer la première vague. Une fois la première vague éliminé, il ne se passe rien tant qu’ils ne sortent pas du vaisseau. Ils ne peuvent pas faire redécoller le vaisseau. S’il ne parviennent pas a éliminer la première vague en 50 tours, les Rakgouls pénaitrent dans le vaisseau, il faudra les finir à la main.

Une fois sorti du vaisseau, la deuxième vague de Rakghouls (prévoir 1.5 Rakghoul par héros) s’élance vers eux tandis que Céleste se contente d’observer de loin. Laissez les héros engager le combat, s’ils ne s’en sortent pas ou quand il ne reste qu’1 ou 2 enemi, lancer une nouvelle vague, massive cette fois avec des \nameref{sec:rakghoul-amblyope} en prime, ça arrive de tout les cotés. Il faut que vos joueurs se sentent perdu. 

\paragraph{Résistance}
Puis quand ils ont bien paniqué, l’Uhumele sort de la couche nuageuse et commence à canarder dans tout les sens sur les Rakghouls. Et là, c’est Dass Jennir qui saute du vaisseau et qui vient se placer au coté des héros, suivit de tout l'équipage de \nameref{sec:schurk-heren}. 
\begin{quotebox}
    \nameref{sec:dass-jennir}: On s’occupe de vous ouvrir la voie, faite ce que vous devez faire !
\end{quotebox}

\paragraph{Empire}
Puis quand ils ont bien paniqué, un Croiseur légé de l’Empire sort de la couche nuageuse et commence à canarder dans tout les sens sur les Rakghouls. Et là, c’est \nameref{sec:garan-keggle} qui saute du vaisseau et qui vient se placer au coté des héros, suivit d’un contingeant de \nameref{sec:storm-trooper}. 
\begin{quotebox}
    \nameref{sec:garan-keggle}: On s’occupe de vous ouvrir la voie, faite ce que vous devez faire !
\end{quotebox}

\paragraph{Commun}
La bataille fait rage, ça part dans tout les sens, de temps à autre faite un lancer de dés, si plus de 4, les héros se trouvent face à face à autant de Rakghouls que le dés dépassent 4 (Ex: 6 = 2 Rakghouls). Ils sont alors obligé de les affronter. 

Après moulte batailles, les héros se retrouvent face à Céleste Morne, possédé par l’esprit de Muur qui s’adresse aux héros (intimidation). Au lieu d'une phrase, Karness semble avoir du mal a parler, il semble se battre contre un démon intérieur ... 
\begin{quotebox}
    \nameref{fig:karness-muur}: Fuyez vous ne pourrez pas le maitriser ... 
\end{quotebox}

C’est alors que commence la phase de combat. A chaque tour, Céleste fait un jet \textit{d’\^Ame}, si le jet est réussit, Karness attaque, si le jet est raté, Céleste parvient à le retenir en cas d’échec critique, Karness est secoué.

Les héros doivent charger une arme (Sabre Laser ou Phaser) avec la Force. S’il n’y a personne de sensible à la Force dans le groupe, faite venir \nameref{sec:dass-jennir} ou \nameref{sec:garan-keggle} avec eux jusqu’à Céleste afin de charger l’arme.

Peu importe qui charge l’arme, il ne pourra rien faire d’autre pendant son tour. Pour charger l’arme il doit réussir 2 jet de \textit{Maîtrise de la Force}.

Ensuite il faudra que quelqu’un (l’un des héros PJ) utilise l’arme et réussisse l’action. S’il touche c’est gagné, s’il rate, il faut recommencer. En attendant, les héros sont confronté à Karness ou à des petits groupe de Rakghouls. Si les héros frappe Céleste trop facilement, faite leur frapper un deuxième coup pour que ça fonctionne.

\subsection{Epilogue}
Le Talisman est touché, Céleste se met à hurler et une vague de Force Lumineuse part de son corps et s’étend sur l’ensemble du cratère. Les Rakghouls touchés par cette vague perdent toute cohésion et combativité. Le Talisman se détache alors du poignet de Céleste et tombe à terre en même temps que Céleste s’écroule au sol, en quelques seconde son corps se met à veillir jusqu’à devenir poussière. Le Talisman la maintenait en vie depuis plus de 1000~ans.

L’oubliette se trouve à 200~m derrière Céleste, c’est à vos héros de voir ce qu’ils font. Mais s’ils ne font rien dans les 2~mn, le Talisman va s’accrocher à \nameref{sec:dass-jennir} ou \nameref{sec:garan-keggle} et dans ce cas tout est perdu. Normalement ils devraient balancer le Talisman dans l’oubliette et refermer.

Bon là c’est la méga-happy-end à vous de voir.

\subsubsection{Progression}
Les héros reçoivent 4~XP pour ce scénario. Ils reçoivent aussi un Atout \textit{Contact} en la personne de \nameref{sec:dass-jennir} ou \nameref{sec:garan-keggle} qui les appréciera pour les qualités dont ils ont fait preuve dans cette mission.
	\include{tex/jokers}
	\include{tex/bestiaire}

	\onecolumn
	\nocite{*}
	\printbibliography
\end{document}